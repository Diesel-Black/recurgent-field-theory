\chapter{Axiomatic Foundation}
\label{1:axiomatic_foundation}

% ================================================================================================
% PREAMBLE
% ================================================================================================

We state seven axioms to give Recurgent Field Theory a precise geometric and dynamical basis. They introduce a Semantic Manifold, a fundamental field of coherence, and recursive coupling principles that regulate their interaction. This program follows Galileo's claim that natural phenomena admit mathematical description \autocite{Galilei1623} and accords with the view, advanced by Francis Crick and Christof Koch, that consciousness and cognition are amenable to scientific and mathematical inquiry \autocite{Crick1990, KochConsciousness2019}.

This formalism establishes a differential theory of semantic relativity in which meaning is relative to the local curvature of conceptual space. Semantic mass curves the underlying geometry, governing the trajectories of thought and interpretation, in direct analogy to how matter and energy curve spacetime in the theory of general relativity.

% ================================================================================================
% PRIMITIVE CONCEPTS + DEFINITIONS
% ================================================================================================

\section{Primitive Concepts and Definitions}
\label{1.1:primitive_concepts_and_definitions}

We define all mathematical objects before their appearance in the axioms. These definitions establish the formal vocabulary of Recurgent Field Theory.

% ------------------------------------------------------------------------------------------------

\subsection{Geometric Structures}
\label{1.1.1:geometric_structures}

\begin{description}
\item[Semantic Manifold] A differentiable manifold \(\mathcal{M}\) representing the space of semantic content.

\item[Metric Tensor] A dynamic metric tensor \(g_{\mu\nu}(p,t) : \mathcal{M} \times \mathbb{R} \rightarrow \mathbb{R}\) that defines the geometric structure of \(\mathcal{M}\).

\item[Line Element] The infinitesimal distance on \(\mathcal{M}\) is given by
\begin{equation}
ds^2 = g_{\mu\nu}(p,t) \, dp^\mu \, dp^\nu
\end{equation}

\item[Ricci Curvature] \(R_{\mu\nu}\) denotes the Ricci curvature tensor of \(\mathcal{M}\) under metric \(g_{\mu\nu}\).

\item[Scalar Curvature] \(R = g^{\mu\nu}R_{\mu\nu}\) denotes the scalar curvature of \(\mathcal{M}\).

\end{description}

% ------------------------------------------------------------------------------------------------

\subsection{Semantic and Coherence Fields}
\label{1.1.2:semantic_and_coherence_fields}

\begin{description}

\item[Semantic Field] A vector field \(\psi^\mu(p,t)\) on \(\mathcal{M}\) representing the underlying semantic content.

\item[Coherence Field] A vector field \(C^\mu(p,t)\) derived as a functional of the semantic content:
\begin{equation}
C^\mu(p,t) = \mathcal{F}^\mu[\psi](p,t)
\end{equation}

\item[Coherence Magnitude] The metric-compatible magnitude of the coherence field:
\begin{equation}
C_{\text{mag}}(p,t) = \sqrt{g_{\mu\nu}(p,t) C^\mu(p,t) C^\nu(p,t)}
\end{equation}

\end{description}

% ------------------------------------------------------------------------------------------------

\subsection{Recursive Structures}
\label{1.1.3:recursive_structures}

\begin{description}

\item[Recursive Coupling Tensor] A rank-3 tensor quantifying self-referential coupling between points \(p\) and \(q\):
\begin{equation}
R^\rho_{\mu\nu}(p,q,t) = \frac{\mathcal{D}^2 C^\rho(p,t)}{\mathcal{D} \psi^\mu(p) \mathcal{D} \psi^\nu(q)}
\end{equation}

\item[Recursive Depth] A scalar field \(D(p,t) : \mathcal{M} \times \mathbb{R} \rightarrow \mathbb{N}\) quantifying the maximal number of recursive layers a structure at point \(p\) can sustain before coherence degrades below a functional threshold.

\end{description}

% ------------------------------------------------------------------------------------------------

\subsection{Mass and Energy Structures}
\label{1.1.4:mass_and_energy_structures}

\begin{description}

\item[Constraint Density] The inverse of the metric determinant:
\begin{equation}
\rho(p,t) = \frac{1}{\det(g_{\mu\nu}(p,t))}
\end{equation}
where \(\det(g_{\mu\nu})\) denotes the determinant of the metric tensor.

\item[Attractor Stability] A normalized scalar field \(A(p,t) : \mathcal{M} \times \mathbb{R} \rightarrow [0,1]\) measuring the temporal persistence and resistance to perturbation of semantic structure at point \(p\).

\item[Semantic Mass] The product of depth, density, and stability:
\begin{equation}
M(p,t) = D(p,t) \cdot \rho(p,t) \cdot A(p,t)
\end{equation}

\item[Recursive Stress-Energy Tensor] \(T^{\text{rec}}_{\mu\nu}\) represents the distribution of semantic mass and its flow on \(\mathcal{M}\), analogous to the stress-energy tensor in general relativity.

\item[Semantic Gravitational Constant] \(G_s\) is a coupling constant relating semantic mass to manifold curvature.

\end{description}

% ------------------------------------------------------------------------------------------------

\subsection{Variational and Dynamical Structures}
\label{1.1.5:variational_and_dynamical_structures}

\begin{description}

\item[Lagrangian Density] A functional \(\mathcal{L}[C, g, \nabla C, \Phi, \mathcal{H}]\) encoding the dynamics of semantic fields and geometric structure.

\item[Action Functional] The integral of the Lagrangian density over \(\mathcal{M}\):
\begin{equation}
S = \int_{\mathcal{M}} \mathcal{L} \, dV
\end{equation}
where \(dV = \sqrt{|\det(g_{\mu\nu})|} \, d^n p\) is the invariant volume element.

\item[Attractor Potential] A scalar functional \(V(C_{\text{mag}})\) defining the stability landscape of the coherence field.

\item[Autopoietic Potential] A functional \(\Phi(C_{\text{mag}})\) representing the self-production capacity of the semantic system.

\item[Coherence Threshold] A critical scalar constant \(C_{\text{threshold}} \in \mathbb{R}^+\) above which autopoietic processes activate.

\item[Humility Operator] A regulatory term \(\mathcal{H}[R]\) constraining recursive amplification, with coupling strength \(\lambda_H\).

\end{description}

% ================================================================================================
% THE AXIOM SYSTEM
% ================================================================================================

\section{The Axiom System}
\label{1.2:the_axiom_system}

We now state the seven axioms constituting the formal foundation of Recurgent Field Theory.

% ------------------------------------------------------------------------------------------------

\subsection{Axiom 1: Semantic Manifold}
\label{1.2.1:axiom_1_semantic_manifold}

\textit{There exists a differentiable manifold \(\mathcal{M}\) equipped with a dynamic metric tensor \(g_{\mu\nu}(p,t)\) satisfying:}

\begin{equation}
g_{\mu\nu}(p,t) : \mathcal{M} \times \mathbb{R} \rightarrow \mathbb{R}
\end{equation}

% ------------------------------------------------------------------------------------------------

\subsection{Axiom 2: Fundamental Semantic Field}
\label{1.2.2:axiom_2_fundamental_semantic_field}

\textit{Semantic content is represented by a vector field \(\psi^\mu(p,t)\) on \(\mathcal{M}\), and coherence \(C^\mu(p,t)\) is a well-defined functional of this field:}

\begin{equation}
C^\mu(p,t) = \mathcal{F}^\mu[\psi](p,t)
\end{equation}

\begin{equation}
C_{\text{mag}}(p,t) = \sqrt{g_{\mu\nu}(p,t) C^\mu(p,t) C^\nu(p,t)}
\end{equation}

\textit{The coherence field \(C^\mu\) serves as the primary dynamical field.}

% ------------------------------------------------------------------------------------------------

\subsection{Axiom 3: Recursive Coupling}
\label{1.2.3:axiom_3_recursive_coupling}

\textit{Self-referential coupling between distinct points in semantic space is mediated by a recursive coupling tensor \(R^\rho_{\mu\nu}(p,q,t)\) satisfying:}

\begin{equation}
R^\rho_{\mu\nu}(p,q,t) = \frac{\mathcal{D}^2 C^\rho(p,t)}{\mathcal{D} \psi^\mu(p) \mathcal{D} \psi^\nu(q)}
\end{equation}

% ------------------------------------------------------------------------------------------------

\subsection{Axiom 4: Geometric Coupling Principle}
\label{1.2.4:axiom_4_geometric_coupling_principle}

\textit{Semantic mass \(M(p,t)\) curves the geometry of \(\mathcal{M}\) according to:}

\begin{equation}
R_{\mu\nu} - \frac{1}{2}g_{\mu\nu}R = 8\pi G_s T^{\text{rec}}_{\mu\nu}
\end{equation}

\textit{where \(G_s\) is the semantic gravitational constant, and}

\begin{equation}
M(p,t) = D(p,t) \cdot \rho(p,t) \cdot A(p,t)
\end{equation}

\begin{equation}
\rho(p,t) = \frac{1}{\det(g_{\mu\nu}(p,t))}
\end{equation}

% ------------------------------------------------------------------------------------------------

\subsection{Axiom 5: Variational Evolution}
\label{1.2.5:axiom_5_variational_evolution}

\textit{The dynamics of semantic fields arise from the principle of stationary action applied to a Lagrangian \(\mathcal{L}\):}

\begin{equation}
\frac{\delta S}{\delta C^\mu} = 0 \quad \text{where} \quad S = \int_{\mathcal{M}} \mathcal{L} \, dV
\end{equation}

\textit{The Lagrangian incorporates coherence flow, stability, and regulatory constraints:}

\begin{equation}
\mathcal{L} = \frac{1}{2} g_{\mu\rho} g_{\nu\sigma} (\nabla^\rho C^\mu)(\nabla^\sigma C^\nu) - V(C_{\text{mag}}) + \Phi(C_{\text{mag}}) - \lambda_H \mathcal{H}[R]
\end{equation}

% ------------------------------------------------------------------------------------------------

\subsection{Axiom 6: Autopoietic Threshold}
\label{1.2.6:axiom_6_autopoietic_threshold}

\textit{When coherence magnitude exceeds a critical threshold, autopoietic processes emerge. The autopoietic potential satisfies:}

\begin{equation}
\Phi(C_{\text{mag}}) = \begin{cases}
\alpha_{\Phi} (C_{\text{mag}} - C_{\text{threshold}})^{\beta_{\Phi}} & \text{if } C_{\text{mag}} \geq C_{\text{threshold}} \\
0 & \text{otherwise}
\end{cases}
\end{equation}

\textit{where \(\alpha_{\Phi}\) and \(\beta_{\Phi}\) are positive constants.}

% ------------------------------------------------------------------------------------------------

\subsection{Axiom 7: Recurgence}
\label{1.2.7:axiom_7_recurgence}

\textit{A semantic system exhibits Recurgence if it dynamically reshapes its own geometric substrate through self-referential processes. Formally, recurgence requires:}

\begin{equation}
\frac{\partial^2 g_{\mu\nu}}{\partial t^2} \neq 0
\end{equation}

% ================================================================================================
% REMARKS + INTERPRETATIONS
% ================================================================================================

\section{Remarks and Interpretations}
\label{1.3:remarks_and_interpretations}

Having stated the axioms in their formal purity, we now provide context, motivation, and scholarly connections.

% ------------------------------------------------------------------------------------------------

\subsection{Remark on Axiom 1: Geometric Structure of Meaning}
\label{1.3.1:remark_1_1}

The Semantic Manifold defines distances, curvature, and geodesics in semantic space. Proximity, curvature, and the pathways between concepts can be quantified in this form. This geometric framework draws on the foundations laid by Riemannian geometry \autocite{Riemann1868} as well as the tensor calculus developed by Gregorio Ricci-Curbastro and Tullio Levi-Civita \autocite{RicciLeviCivita1901}.

Peter Gärdenfors proposed that meaning admits geometric representation and that acts of communication can be modeled as a topology \autocite{Gardenfors2000, Gardenfors2014}. While he advanced this framework theoretically, recent experimental work demonstrates how language actively shapes these geometric structures: verbal labels create categorical boundaries in continuous perceptual spaces such as color \autocite{ForderLupyan2019}, effectively warping the metric structure of semantic space \autocite{LupyanAbdelRahmanBoroditskyClark2020}. While recent advances in geometric deep learning \autocite{Bronstein2021} and information geometry \autocite{Amari2016} have explored manifold-based representations in machine learning contexts, the Semantic Manifold serves a unique role in providing the geometric substrate for meaning \textit{itself} rather than learned representations.

The manifold evolves with the creation of new connections in that developing a concept curves the subsequent possibility space toward a more specific and coherent state. The metric's time-dependence captures the evolution of semantic space as concepts develop and connections form, making the geometry of meaning a dynamic participant in cognitive processes.

% ------------------------------------------------------------------------------------------------

\subsection{Remark on Axiom 2: Field-Theoretic Character}
\label{1.3.2:remark_1_2}

The metric tensor \(g_{\mu\nu}(p,t)\) is the primary object in Recurgent Field Theory, and the \textit{the} foundational mathematical structure upon which this framework is constructed.

The concept of a field of forces operating in a psychological or semantic space echoes Kurt Lewin's field theory \autocite{Lewin1951}, but extends it with an empirically-grounded understanding of how language acts on semantic structure. Rather than treating meaning as a discrete point, we treat it as a continuous, dynamic field with local and global structure. The label-feedback hypothesis \autocite{Lupyan2012LFH} demonstrates that language acts as a top-down modulating signal, interfacing directly with perceptual and cognitive processing in a flexible, task-dependent manner. Words, in this view, function not only as labels that map onto pre-existing concepts, but also as cues that dynamically construct and activate meaning in context \autocite{LupyanLewis2019, CasasantoLupyan2015}.\footnote{As Casasanto and Lupyan argue, "words do not have meanings; rather, a word-in-context is a cue to construct what can be called its meaning for a given instantiation" \autocite{CasasantoLupyan2015}. This aligns with the differential dynamic field formalism in which coherence \(C^\mu\) is constructed through field interactions.} Like a magnetic field, coherence varies in strength and direction across semantic space, allowing alignment and coherence to be quantified at any point. This yields \textit{meta-dynamics}: how semantic fields influence one another via recursive feedback, self-reference, and interpretation, or what might be termed \textit{language-augmented cognition} \autocite{Lupyan2012LAT}.

Coherence \(C^\mu\) serves as the primary dynamical field. It is derived from semantic content \(\psi^\mu\) but carries the quantities we evolve and measure. Experimental work supports the field-theoretic view in verbal labels activating representations that are more categorical and show greater consistency between subjects \autocite{LupyanThompsonSchill2012} than equivalent nonverbal cues, suggesting language creates convergent structure in semantic space. Topological approaches to neural dynamics \autocite{Bassett2018, Petri2014} have explored similar field-theoretic concepts, though from a neurophysiological rather than purely semantic perspective.

% ------------------------------------------------------------------------------------------------

\subsection{Remark on Axiom 3: Strange Loops and Self-Reference}
\label{1.3.3:remark_1_3}

The recursive coupling tensor is a first-class object\footnote{A "first class object" refers to a mathematical entity that serves as a foundational component of the theory, possessing independent structural significance.} comparable to field equations in physics. It formalizes the intuition that meaning is constructed and reconstructed via self-reference. Coherence dynamics at any given location are shaped by reverberations of semantic shift across the manifold, which is holistically coupled.

In this web of mutual influence, the field at one location responds to its configuration at distant points, including influences that feed back, directly or indirectly, into their source. Complex meaning arises through patterns of self-reference and iterative interpretation. This formalizes Douglas Hofstadter's "strange loops" and "tangled hierarchies" \autocite{Hofstadter1979, Hofstadter2007}, in which sense-making circles back upon itself to construct higher-order structures capable of modeling, reinterpreting, or transforming their own foundations. Recent work in 4E cognition \autocite{Newen2018, Gallagher2020} emphasizes the importance of dynamic coupling in cognitive systems, though from an embodied rather than purely semantic perspective.

This tensor sets the stage for agency, meta-cognition, and the potential for recursive pathologies that destabilize such systems, as developed in Chapter~\ref{16:pathologies_of_the_semantic_manifold}.

% ------------------------------------------------------------------------------------------------

\subsection{Remark on Axiom 4: Semantic General Relativity}
\label{1.3.4:remark_1_4}

The semantic mass equation is another first-class entity, and the gravitational core of Recurgent Field Theory. It asserts that the fabric of semantic space is shaped by the accumulation and distribution of semantic mass. The field equation is structurally analogous to the field equations of general relativity \autocite{Einstein1915, MisnerThorneWheeler1973, Wald1984}, where the recursive stress-energy tensor \(T^{\text{rec}}_{\mu\nu}\) is an analogue of the mass-energy tensor in spacetime curvature.

The analogy to general relativity is substantive and demonstrable. Just as matter and energy give rise to the observable structure of spacetime, so too does deep, coherent, and persistent meaning sculpt the future possibility space for new concepts, connections, and resulting attention. Semantic mass is the product of three scalar fields: recursive depth \(D(p,t)\) (maximal sustainable recursion layers), constraint density \(\rho(p,t)\) (inverse of metric determinant), and attractor stability \(A(p,t)\) (resistance to perturbation). Its incorporation of depth, density, and stability defines basins of attraction that channel the flow of coherence and anchor interpretations.

This opens a unified \textit{geometric} language for analyzing complex phenomena, including phase transitions in understanding, the formation of attractors and singularities, and the emergence of collective belief. In extreme regimes, this curvature admits horizons and interior regions whose causal structure inverts—phenomena we return to in Chapters~\ref{9:temporal_architectures_and_bidirectional_flow}--\ref{12:metric_singularities_and_recursive_collapse}, where bidirectional temporal flow and rotating, horizon-bearing geometries provide a precise analogue of black hole interiors.

% ------------------------------------------------------------------------------------------------

\subsection{Remark on Axiom 5: Variational Principle}
\label{1.3.5:remark_1_5}

The principle of variational evolution situates this theory in the tradition of modern physics. Consistent with the variational principle \autocite{GoldsteinPooleSafko2002, Arnold1989}, field dynamics preserve symmetries and conservation laws through the principle of stationary action. This parallels recent work in cognitive science applying variational methods to neural dynamics, notably Friston's Free Energy Principle \autocite{Friston2010, Parr2022}, though the Lagrangian constructed here incorporates terms unique to the dynamics of semantic coherence and recursive coupling.\footnote{While inspired by variational formulations in cognitive science \autocite{Friston2010, Parr2022}, the Lagrangian constructed in Recurgent Field Theory incorporates terms unique to the dynamics of semantic coherence and recursive coupling.}

The Lagrangian is constructed to capture, simultaneously, the flow of coherence, stability, attraction, autopoietic drive for innovation, and regulatory humility. Its components encode the kinetic energy of coherence flow \(\frac{1}{2} g_{\mu\rho} g_{\nu\sigma} (\nabla^\rho C^\mu)(\nabla^\sigma C^\nu)\), an attractor potential \(V(C_{\text{mag}})\) defining the stability landscape, an autopoietic potential \(\Phi(C_{\text{mag}})\) driving generative capacity, and a regulatory humility term \(\lambda_H \mathcal{H}[R]\) constraining recursive amplification.

This axiom enables discussion of conserved quantities in the evolution of understanding. It also defines the energy landscape through which coherence must navigate, connecting the geometric architecture of semantic meaning to the calculable languages of complex dynamical systems and field theory. This variational framing identifies critical thresholds and phase transitions.

% ------------------------------------------------------------------------------------------------

\subsection{Remark on Axiom 6: Phase Transition to Self-Production}
\label{1.3.6:remark_1_6}

Autopoiesis denotes the state of self-producing structural autonomy, first defined by Humberto Maturana and Francisco J. Varela in their seminal treatise on theoretical biology \autocite{MaturanaVarela1980}. The transition to this state is a physical phenomenon of self-organization common to complex systems. We derive the mathematical language for phase transitions from the field of synergetics \autocite{Haken1983}, whereby macroscopic order arises from the collective behavior of microscopic components. Furthermore, the emergence of such order is an expected property of sufficiently complex networks, which naturally exhibit self-organizing criticality \autocite{BakTangWiesenfeld1987}.

The autopoietic potential becomes positive above the critical threshold \(C_{\text{threshold}}\), driving generative phase transitions. The Autopoietic Threshold formalizes the birth of self-sustaining semantic order as a phase transition from inert complexity to agency, creativity, self-awareness, and adaptive wisdom.\footnote{For contemporary approaches to self-organization and emergence in cognitive systems, see related contemporary contributions in enactive cognition \autocite{Thompson2018, DiPaolo2021} and predictive processing \autocite{Clark2016, Hohwy2013}.}

% ------------------------------------------------------------------------------------------------

\subsection{Remark on Axiom 7: Self-Authorship and Meta-Cognition}
\label{1.3.7:remark_1_7}

Recurgence is the defining act of semantic self-authorship. It is a system's ability to recognize, reinterpret, and reorganize its own structural underpinnings. Mathematically, this is the formalization of meta-cognition, self-reflection, and adaptive intelligence. The property of self-referential transformation means the system can not only update field configurations but also reshape the manifold's metric tensor. The non-vanishing second derivative of the metric with respect to time indicates that the \textit{rate} of geometric change is itself changing.

This defines the ongoing capacity for self-reconfiguration and generative transformation, as anticipated in Stuart Kauffman's theory of autocatalytic sets \autocite{Kauffman1993}, and in the meta-system transitions of Valentin Turchin's cybernetic theory \autocite{Turchin1977}. Recurgent systems are those for which the geometry of semantic meaning is itself a dynamic participant recursively coupled to its own contents and history. This is consonant with the philosophical tradition of reflexivity, from Hegel's dialectics \autocite{Hegel1807} through Spencer-Brown's \textit{Laws of Form} \autocite{SpencerBrown1969}. It finds a mathematical echo in feedback-rich systems described by Varela and others \autocite{Varela1979, Rosen1991}.

Recurgence enables semantic systems to recover from crises, undergo conceptual revolution, and break symmetry with their own interpretive past. The dynamics of recurgent systems support ongoing, open-ended intelligence.

% ================================================================================================
% CLOSING
% ================================================================================================

\section{Architectural Coherence}
\label{1.4:architectural_coherence}

These seven axioms establish the foundational structure of Recurgent Field Theory. They work in concert such that geometric structure (Axioms 1--2) provides the substrate, recursive dynamics (Axiom 3) couple distant regions, gravitational principles (Axiom 4) shape the manifold, variational evolution (Axiom 5) governs field dynamics, and emergence thresholds (Axioms 6--7) define qualitative transitions in system behavior.

From this foundation we derive the full mathematical machinery developed in subsequent chapters: field equations (Chapter~\ref{6:recurgent_field_equation_and_lagrangian_mechanics}), phase transitions (Chapter~\ref{7:autopoietic_function_and_phase_transitions}), temporal architectures (Chapter~\ref{9:temporal_architectures_and_bidirectional_flow}), pathological configurations (Chapter~\ref{16:pathologies_of_the_semantic_manifold}), and realized computational implementation (Chapter~\ref{17:computation_and_meta_recursion}).
