\chapter{Semantic Manifold and Metric Geometry}
\label{3:semantic_manifold_and_metric_geometry}

% ------------------------------------------------------------------------------------------------

\section{Overview}
\label{3.1:overview}

We establish the geometric foundation of Recurgent Field Theory as a differentiable Semantic Manifold, \(\mathcal{M}\), the structure of which encodes the complete configuration space of \textit{meaning}. This concept has historical parallels to the abstract state spaces of modern physics \autocite{vonNeumann1932}, and is formally embeddable in Euclidean space for analysis \autocite{Whitney1936}. The manifold's metric tensor, \(g_{\mu\nu}(p, t)\), evolves with semantic processes to create a dynamic landscape of conceptual "distance" and curvature. In high-constraint regions, the geometry is rigid and confines thought to well-defined paths. In low-constraint regions, the geometry is fluid and permits innovation. Semantic mass curves this geometry, a quantity derived from the depth, density, and stability of meaningful semantic structures. The resulting curvature governs the formation of attractor basins, future interpretation, and subsequent structural formation.

% ------------------------------------------------------------------------------------------------

\section{The Metric Tensor and Semantic Distance}
\label{3.2:the_metric_tensor_and_semantic_distance}

The intrinsic curvature of semantic space cannot be captured by static Euclidean geometry, as the cognitive effort required to move between ideas varies systematically. The representation of psychological or conceptual similarity by distance in a metric space has precedent in mathematical psychology \autocite{Shepard1987}. To formalize and account for this variance, we draw on Riemannian geometry \autocite{Riemann1868, doCarmo1992}, employing a dynamic metric tensor, \(g_{\mu\nu}(p,t)\), which evolves as semantic structures form and decay. Here, we adopt that principle, positing that the metric tensor provides the structure for such a space.

The infinitesimal squared distance \(ds^2\) between two neighboring points in semantic space is given by:

\begin{equation}
ds^2 = g_{\mu\nu}(p, t) \, dp^\mu \, dp^\nu
\end{equation}

where \(dp^\mu\) represents an infinitesimal displacement. The metric \(g_{\mu\nu}\) encodes the local constraint structure of \textit{meaning} and modulates the cost of semantic displacement. High values of its components correspond to regions where semantic distinctions are rigid; low values mark regions of semantic fluidity.

% ------------------------------------------------------------------------------------------------

\section{Evolution Equation for the Semantic Metric}
\label{3.3:evolution_equation_for_the_semantic_metric}

A flow equation analogous to Ricci flow \autocite{Hamilton1982, Perelman2002, RicciLeviCivita1901} governs the metric tensor's evolution, but with added forcing terms reflecting the influence of recursive structure. This equation specifies the deformation of semantic geometry under both its intrinsic curvature, as well as that of feedback from nonlocal processes:

\begin{equation}\label{eq:metric_evolution}
\frac{\partial g_{\mu\nu}}{\partial t} = -2 R_{\mu\nu} + F_{\mu\nu}(R, D, A)
\end{equation}

where \(R_{\mu\nu}\) is the Ricci curvature tensor of \(g_{\mu\nu}\). The forcing term \(F_{\mu\nu}\) is a symmetric tensor-valued functional of the recursive coupling tensor \(R\), the recursive depth field \(D\), and the attractor stability field \(A\).

% ------------------------------------------------------------------------------------------------

\section{Constraint Density}
\label{3.4:constraint_density}

The metric tensor determines the constraint density \(\rho(p, t)\) at each point on the manifold:

\begin{equation}
\rho(p, t) = \frac{1}{\det(g_{\mu\nu}(p, t))}
\end{equation}

High constraint density (\(\rho \gg 1\)) corresponds to tightly packed semantic states where transitions are suppressed. Conversely, low-density regions (\(\rho \ll 1\)) mark areas of semantic flexibility where innovation is energetically favorable.

% ------------------------------------------------------------------------------------------------

\section{The Coherence Field}
\label{3.5:the_coherence_field}

The coherence field \(C^\mu(p, t)\) is a vector field on \(\mathcal{M}\) that represents the local alignment and self-consistency of semantic structures. Its metric defines the field's scalar magnitude, quantifying the total strength of coherence at a point, independent of direction:

\begin{equation}
C_{\mathrm{mag}}(p, t) = \sqrt{g_{\mu\nu}(p, t) C^\mu(p, t) C^\nu(p, t)}
\end{equation}

where \(g^{\mu\nu}\) is the inverse metric. This provides us with the basis for defining the attractor and autopoietic potentials in subsequent chapters.

% ------------------------------------------------------------------------------------------------

\section{Recursive Depth, Attractor Stability, and Semantic Mass}
\label{3.6:recursive_depth_attractor_stability_and_semantic_mass}

Scalar fields for recursive depth, \(D(p, t)\), and attractor stability, \(A(p, t)\), modulate the manifold's geometry. The depth \(D\) quantifies the maximal recursion a structure at \(p\) can sustain before its coherence degrades, while stability \(A\) measures its resilience to perturbation. Together with the constraint density \(\rho\), these fields compose the semantic mass:

\begin{equation}
M(p, t) = D(p, t) \cdot \rho(p, t) \cdot A(p, t)
\end{equation}

Semantic mass \(M(p,t)\) curves the manifold, generating attractor basins and shaping the flow of coherence. High-mass regions are strong attractors that anchor interpretation, while low-mass regions are more amenable to recursive innovation. 